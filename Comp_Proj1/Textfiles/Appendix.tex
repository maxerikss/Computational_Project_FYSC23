\section{The code}
\subsection{First problem}
\begin{verbatim}
import numpy as np
import matplotlib.pyplot as plt

#!/usr/bin/env python3
# -*- coding: utf-8 -*-
"""
Created on Mon Feb 10 10:28:49 2025

@author: fredrik
"""

N = 501         # Chain length
V = -1          # Hopping term
epsilon = 0     # Energy at site
gamma = 0.05    # Broadening factor
LDOS_sites = [1, 2, 3, 5, 10, 50, 100, 251] # Sites we want to plot
energy_range = np.linspace(-6, 6, 500)  # Energy range for LDOS calculation


def hamiltonian(n, epsilon, V):
    "Create a hamilitonian"
    upper =  np.diag(V * np.ones(n-1), 1)
    middle = np.diag(epsilon * np.ones(n), 0)
    lower = np.diag(V * np.ones(n-1), -1)
    return upper + middle + lower

def sums(gamma, eigenvec, lamb, energy, eigenergy, site):
    "Each term in the sum"
    return (gamma / np.pi) * (eigenvec[site-1, lamb] ** 2) / ((energy - eigenergy[lamb])**2 + gamma**2)
    
# Find the hamltonian
H = hamiltonian(N, epsilon, V)

# Find eigenergies and eigenvectors
eigenenergies, eigenvectors = np.linalg.eigh(H)

# initlaize the LDOS as a dictonary
LDOS = {site: np.zeros(len(energy_range)) for site in LDOS_sites}

# Do the calculation for each site which we are intrested in
for site in LDOS_sites:
    # Make a loop where we look through each position with the energy at that poition 
    for i, E in enumerate(energy_range):
        # Calculate the sum
        for lamb in range(N):
            LDOS[site][i] += sums(gamma=gamma, eigenvec=eigenvectors, 
                                  lamb=lamb, energy=E, 
                                  eigenergy=eigenenergies, site=site)

# Plot LDOS for the selected sites
plt.figure(figsize=(10, 6))
for site in LDOS_sites:
    plt.plot(energy_range, LDOS[site], label=f"i={site}")
plt.xlabel("Energy E")
plt.ylabel(r"$g^L_{i}(E, \Gamma)$")
plt.title("Local Density of States for Selected Sites")
plt.legend()
plt.grid()

plt.savefig("Comp_Proj1/Figures/task1.pdf")
plt.show()
\end{verbatim}


\subsection{Second problem}
\begin{verbatim}
    #!/usr/bin/env python3
    # -*- coding: utf-8 -*-
    """
    Created on Mon Feb 10 10:28:49 2025
    
    @author: fredrik
    """
    
    import numpy as np
    import matplotlib.pyplot as plt
    
    N = 501         # Chain length
    V = -1          # Hopping term
    epsilon = 0     # Energy at site
    gamma = 0.05    # Broadening factor
    LDOS_sites = [0, 1, 2, 3, 5, 10, 50, 100, 251] # Sites we want to plot
    energy_range = np.linspace(-6, 6, 1000)  # Energy range for LDOS calculation
    
    def hamiltonian(n, epsilon, V, e0, v0):
        "Create a Hamiltonian"
        upper = np.diag(V * np.ones(n-1), 1)
        middle = np.diag(epsilon * np.ones(n), 0)
        lower = np.diag(V * np.ones(n-1), -1)
        H = upper + middle + lower
        H[0, -1] = v0
        H[-1, 0] = v0
        H[-1, -1] = e0
        return H
    
    def sums(gamma, eigenvec, lamb, energy, eigenergy, site):
        "Each term in the sum"
        return (gamma / np.pi) * (eigenvec[site-1, lamb] ** 2) / ((energy - eigenergy[lamb])**2 + gamma**2)
    
    def compute_LDOS(LDOS_sites, energy_range, H):
        "Function for fining the LDOS"
        # Find eigenenergies and eigenvectors
        eigenenergies, eigenvectors = np.linalg.eigh(H)
    
        # Initialize LDOS as a dictionary
        LDOS = {site: np.zeros(len(energy_range)) for site in LDOS_sites}
        
        # Compute LDOS for each site of interest
        for site in LDOS_sites:
            for i, E in enumerate(energy_range):
                for lamb in range(N):
                    LDOS[site][i] += sums(gamma=gamma, eigenvec=eigenvectors, 
                                          lamb=lamb, energy=E, 
                                          eigenergy=eigenenergies, site=site)
        return LDOS # Return the LDOS
    
    # Different parameter sets (e0, v0)
    param_list = [(-1, 0), (-1, -0.3), (-1, -3)]
    
    # Create figure with 3 subplots
    fig, axes = plt.subplots(3, 1, figsize=(8, 9))
    fig.suptitle("Local Density of States for Selected Sites")
    
    # Loop over different parameter sets and plot in subplots
    for idx, (e0, v0) in enumerate(param_list):
        H = hamiltonian(N, epsilon, V, e0, v0)  # Compute Hamiltonian
        LDOS = compute_LDOS(LDOS_sites, energy_range, H)  # Compute LDOS
        
        for site in LDOS_sites:
            axes[idx].plot(energy_range, LDOS[site], label=f"i={site}")
        
        axes[idx].set_title(f"$\epsilon_0={e0}, V_0={v0}$")
        axes[idx].set_ylabel(r"$g^L_{i}(E, \Gamma)$")
        axes[idx].legend()
        axes[idx].grid()
    
    # Set common x-axis
    axes[0].set_ylim(0, 1.25)
    axes[1].set_ylim(0, 1.5)
    axes[2].set_ylim(0, 3.5)
    axes[2].set_xlabel("Energy E")
    plt.tight_layout()
    
    plt.savefig("Comp_Proj1/Figures/task2.pdf")
    plt.show()    
\end{verbatim}

\subsection{The third problem}
\begin{verbatim}
    #!/usr/bin/env python3
    # -*- coding: utf-8 -*-
    """
    Created on Mon Feb 10 10:51:38 2025
    
    @author: fredrik
    """
    
    import numpy as np
    import matplotlib.pyplot as plt
    from scipy.sparse import csr_matrix
    import time  
    
    start_time = time.time()
    
    
    Nl = 81        # Lattice size along one dimension
    Ns = Nl**2     # Total number of atoms
    V = -1         # Hopping parameter
    epsilon = 0    # On-site energy
    gamma = 0.05   # Broadening factor
    
    Nh = (Nl + 1) // 2
    Nhp = (Nl - 1) // 2
    
    # Sites of interest for LDOS calculation
    LDOS_sites = [(0, 0), (0, 1), (1, 1), (2, 0), (2, 1), (2, 2)]
    energy_range = np.linspace(-8, 8, 200)
    
    
    def coord_to_index(i, j):
        "Convert (i', j') coordinates to matrix index m."
        "This was done by the method provided"
        ibar = i + Nhp 
        jbar = j + Nhp
        return Nl*ibar + jbar
    
    
    def get_neighbors(i, j):
        "This function check if there are any neighbors nearby that exsists"
        # List all possible nearest neighbors (up, down, left, right)
        neighbors = [(i-1,j), (i+1,j), (i,j-1), (i,j+1)]   
        
        valid_neighbors = []
    
        # Check each potential neighbor to ensure it is within the valid range
        for ni, nj in neighbors:
            if -Nhp <= ni <= Nhp and -Nhp <= nj <= Nhp:
                valid_neighbors.append((ni, nj))  
    
        return valid_neighbors # Return the list of valid nearest neighbors
    
    
    def hamiltonian(Ns, epsilon, V):
        "Construct the Hamiltonian matrix for a 2D square lattice."
        H = csr_matrix((Ns, Ns), dtype=complex).tolil()
        H.setdiag([epsilon] * (Ns+1))    
        for i in range(-Nhp, Nhp + 1):
            for j in range(-Nhp, Nhp + 1):
                m = coord_to_index(i, j)
                for ni, nj in get_neighbors(i, j):
                    n = coord_to_index(ni, nj)
                    H[m, n] = V  # Nearest-neighbor hopping
        return H.tocsr()
    
    
    def sums(gamma, eigenvecs, lamb, energy, eigvals, site):
        "Compute each term in the LDOS sum."
        return (gamma / np.pi) * (np.abs(eigenvecs[site, lamb]) ** 2) / ((energy - eigvals[lamb])**2 + gamma**2)
    
    
    def compute_LDOS(LDOS_sites, energy_range, H):
        "Compute the Local Density of States (LDOS)."
    
        # Find eigenenergies and eigenvectors
        eigenenergy, eigenvectors = np.linalg.eigh(H.toarray()) # Convert sparse matrix to dense calculation
        
        # Initialize LDOS as a dictionary
        LDOS = {site: np.zeros(len(energy_range)) for site in LDOS_sites}
        
        # Compute LDOS for each site of interest
        for site in LDOS_sites:
            site_index = coord_to_index(site[0], site[1])
            for i, E in enumerate(energy_range):
                for lamb in range(Ns):
                    LDOS[site][i] += sums(gamma=gamma, eigenvecs=eigenvectors, 
                                          lamb=lamb, energy=E, 
                                          eigvals=eigenenergy, site=site_index)
        return LDOS 
    
    # Compute Hamiltonian and LDOS
    H = hamiltonian(Ns, epsilon, V)
    LDOS = compute_LDOS(LDOS_sites, energy_range, H)
    
    # Plot LDOS
    plt.figure(figsize=(8, 6))
    for site in LDOS_sites:
        plt.plot(energy_range, LDOS[site], label=f"Site {site}")
    plt.xlabel("Energy E")
    plt.ylabel(r"$g^L_{i}(E, \Gamma)$")
    plt.title(f"Local Density of States for Selected Sites with $\epsilon={epsilon}, V={V}$")
    plt.legend()
    plt.grid()
    
    plt.savefig("Comp_Proj1/Figures/task3.pdf")
    plt.show()
    
    
    end_time = time.time()
    print(f"Total time taken: {np.round((end_time - start_time)/60,1)} minutes")    
\end{verbatim}

\subsection{Forth problem}
\begin{verbatim}
    #!/usr/bin/env python3
    # -*- coding: utf-8 -*-
    """
    Created on Mon Feb 10 11:29:10 2025
    
    @author: fredrik
    """
    
    import numpy as np
    import matplotlib.pyplot as plt
    from scipy.sparse import csr_matrix
    import time  
    
    
    Nl = 81                                 # Lattice size along one dimension
    Ns = Nl**2     
    Nh = (Nl+1) // 2
    Nhp = (Nl-1) // 2
    energy_range = np.linspace(-8, 8, 1000)  # Energy range
    V = -1                                  # Hopping parameter
    epsilon = 0                             # On-site energy
    gamma = 0.05                            # Broadening factor
    
    
    def coord_to_index(i, j, Nhp=Nhp, Nl=Nl):
        """ Convert (i', j') coordinates to matrix index m. """
        ibar = i + Nhp
        jbar = j + Nhp
        return Nl*ibar + jbar
    
    def get_neighbors(i, j):
        "This function check if there are any neighbors nearby that exsists"
        # List all possible nearest neighbors (up, down, left, right)
        neighbors = [(i-1,j), (i+1,j), (i,j-1), (i,j+1)]   
        
        valid_neighbors = []
    
        # Check each potential neighbor to ensure it is within the valid range
        for ni, nj in neighbors:
            if -Nhp <= ni <= Nhp and -Nhp <= nj <= Nhp:
                valid_neighbors.append((ni, nj))  
    
        return valid_neighbors # Return the list of valid nearest neighbors
    
    
    def hamiltonian(Ns, epsilon, V):
        "Construct the Hamiltonian matrix for a 2D square lattice."
        H = csr_matrix((Ns, Ns), dtype=complex).tolil()
        for i in range(-Nhp, Nhp + 1):
            for j in range(-Nhp, Nhp + 1):
                m = coord_to_index(i, j)
                H[m, m] = epsilon  # On-site energy
                for ni, nj in get_neighbors(i, j):
                    n = coord_to_index(ni, nj)
                    H[m, n] = V  # Nearest-neighbor hopping
        return H.tocsr()
    
    
    def hamiltonian_adsorbate(Ns, epsilon, V, adsorbate_type, epsilon_0, V_0, Nhp=Nhp, Nl=Nl):
        """ Construct the Hamiltonian matrix including adsorbate effects """
        H = csr_matrix((Ns+1, Ns+1)).tolil()
        
        # Fill the original surface Hamiltonian
        for i in range(-Nhp, Nhp + 1):
            for j in range(-Nhp, Nhp + 1):
                m = coord_to_index(i, j, Nhp, Nl)
                H[m, m] = epsilon  # On-site energy
                for ni, nj in get_neighbors(i, j):
                    n = coord_to_index(ni, nj, Nhp, Nl)
                    H[m, n] = V  # Nearest-neighbor hopping
        
        # Define adsorbate interaction
        adsorbate = Ns  # The additional adsorbate index
        H[adsorbate, adsorbate] = epsilon_0  # On-site energy of adsorbate
        
        if adsorbate_type == "atop":
            m = coord_to_index(0, 0)
            H[m, adsorbate] = V_0
            H[adsorbate, m] = V_0
        
        elif adsorbate_type == "bridge":
            m1 = coord_to_index(0, 0)
            m2 = coord_to_index(1, 0)
            H[m1, adsorbate] = V_0 / np.sqrt(2)
            H[m2, adsorbate] = V_0 / np.sqrt(2)
            H[adsorbate, m1] = V_0 / np.sqrt(2)
            H[adsorbate, m2] = V_0 / np.sqrt(2)
        
        elif adsorbate_type == "center":
            sites = [(0, 0), (1, 0), (0, 1), (1, 1)]
            for i, j in sites:
                m = coord_to_index(i, j, Nhp, Nl)
                H[m, adsorbate] = V_0 / 2
                H[adsorbate, m] = V_0 / 2
        
        elif adsorbate_type == "impurity":
            m = coord_to_index(0, 0, Nhp, Nl)
            H[m, m] = epsilon_0  # Replace surface atom energy
            for ni, nj in get_neighbors(0, 0):
                n = coord_to_index(ni, nj, Nhp, Nl)
                H[m, n] = V_0
                H[n, m] = V_0
        
        return H.tocsr()
    
    def sums(gamma, eigenvecs, lamb, energy, eigvals, site):
        "Compute each term in the LDOS sum."
        return (gamma / np.pi) * (np.abs(eigenvecs[site, lamb]) ** 2) / ((energy - eigvals[lamb])**2 + gamma**2)
    
    
    def compute_LDOS(H, energy_range, Ns=Ns, center=False):
        "Compute the Local Density of States (LDOS)."
        
        # convert to dense array
        H = H.toarray()   
    
        # Find eigenenergies and eigenvectors
        eigenenergy, eigenvectors = np.linalg.eigh(H)  
    
        # Initialize LDOS as a dictionary
        LDOS = np.zeros(len(energy_range))
        
        #Check position of paticle
        if center == True:
            site_index = coord_to_index(0, 0)
        else:
            site_index = Ns
    
        for i, E in enumerate(energy_range):
                for lamb in range(Ns):
                    LDOS[i] += sums(gamma=gamma, eigenvecs=eigenvectors, 
                                          lamb=lamb, energy=E, 
                                          eigvals=eigenenergy, site=site_index)
        return LDOS
    
    #%%
    
    start_time = time.time()
    
    
    # Clean surface LDOS
    clean = compute_LDOS(hamiltonian(Ns, epsilon, V), 
                         energy_range, center=True)  
    
    # Adsorbate cases 1
    atop1 = compute_LDOS(hamiltonian_adsorbate(Ns, epsilon, V, "atop", -2, -1.3),
                         energy_range)
    
    bridge1 = compute_LDOS(hamiltonian_adsorbate(Ns, epsilon, V, "bridge", -2, -1.3),
                         energy_range)
    
    center1 = compute_LDOS(hamiltonian_adsorbate(Ns, epsilon, V, "center", -2, -1.3),
                         energy_range)
    
    impurity1 =  compute_LDOS(hamiltonian_adsorbate(Ns, epsilon, V, "impurity", -2, -1.3),
                         energy_range, center=True)
    
    # Adsorbate cases 2
    atop2 = compute_LDOS(hamiltonian_adsorbate(Ns, epsilon, V, "atop", -2, -5),
                         energy_range)
    
    bridge2 = compute_LDOS(hamiltonian_adsorbate(Ns, epsilon, V, "bridge", -2, -5),
                         energy_range)
    
    center2 = compute_LDOS(hamiltonian_adsorbate(Ns, epsilon, V, "center", -2, -5),
                         energy_range)
    
    impurity2 =  compute_LDOS(hamiltonian_adsorbate(Ns, epsilon, V, "impurity", -2, -5),
                         energy_range, center=True)
    
    
    end_time = time.time()
    print(f"Total time taken: {np.round((end_time - start_time)/60,1)} minutes")
    
    
    
    #%%
    
    fig, axes = plt.subplots(2, 1, figsize=(8, 8))
    fig.suptitle("Local Density of States With Adsorbates for Site (0,0)")
    
    # For the first case 
    axes[0].set_title(r"$\epsilon_0=-1, V_0=-1.3$")
    
    axes[0].plot(energy_range, clean, label="clean", c='black', linestyle=':')
    axes[0].plot(energy_range, atop1, label="atop")
    axes[0].plot(energy_range, bridge1, label="bridge")
    axes[0].plot(energy_range, center1, label="center")
    axes[0].plot(energy_range, impurity1, label="impurity")
    
    axes[0].set_ylabel(r"$g^L_{i}(E, \Gamma)$")
    axes[0].legend()
    axes[0].grid()
    #axes[0].set_ylim(0,0.4)
    
    # For the second case 
    axes[1].set_title(r"$\epsilon_0=-1, V_0=-5$")
    
    axes[1].plot(energy_range, clean, label="clean", c='black', linestyle=':')
    axes[1].plot(energy_range, atop2, label="atop")
    axes[1].plot(energy_range, bridge2, label="bridge")
    axes[1].plot(energy_range, center2, label="center")
    axes[1].plot(energy_range, impurity2, label="impurity")
    
    axes[1].set_ylabel(r"$g^L_{i}(E, \Gamma)$")
    axes[1].legend()
    axes[1].grid()
    #axes[1].set_ylim(0,0.4)
    
    
    # Set common x-axis
    axes[1].set_xlabel("Energy E")
    plt.tight_layout()
    
    plt.savefig("Comp_Proj1/Figures/task4.pdf")
    plt.show()    
\end{verbatim}